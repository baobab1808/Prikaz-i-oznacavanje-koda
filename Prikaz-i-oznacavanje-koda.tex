\documentclass{beamer}
\usetheme{Boadilla}
\usecolortheme{seahorse}
\usefonttheme{structureitalicserif}

\usepackage[croatian]{babel} 
\usepackage[utf8]{inputenc} 
\usepackage{listings}
%\bibliography{bibliography.bib}
\setcounter{tocdepth}{2}

\title{Prikaz i označavanje koda}
\subtitle{Paketi za prikaz i označavanje kodova, pseudo-kodova i algoritama različitih programskih jezika.}
\author{Lucija Žužić i  Barbara Breš}
\institute{Sveučilište u Rijeci, Tehnički fakultet}
\date{\today}

\begin{document}
 
	\begin{frame}	
		\titlepage	
	\end{frame}	

	\begin{frame}
		\frametitle{Sadržaj}
		\tableofcontents
	\end{frame}

	\section{Paket listings}	

	\begin{frame}[fragile]
		\frametitle{Značajke paketa i kompatibilnost}
		\begin{itemize}
			\begin{item}
				 Onemogućuje dodatno procesiranje teksta kao Latex naredbi, tekst se ispisuje kako je unesen  
			\end{item}
			\begin{item}
				Podržava većinu najčešćih programskih jezika, pruža mnogo opcija prilagodbe
			\end{item}
			\begin{item}
				Označava ključne riječi jezika drugačijom bojom, podebljanim ili ukošenim tekstom, broji linije koda
			\end{item}
			\begin{item}	
				Za korištenje moramo instalirati paket listings s interneta (pomoću package managera kojeg koristimo uz Latex editor) i u preambuli dokumenta moramo napisati \verb|\usepackage{listings}| 
			\end{item}
		\end{itemize}
		\begin{alertblock}{Kompatibilnost s beamer paketom}
				Ako koristimo neprocesirani tekst (listings paket i drugi) zajedno s beamer paketom (prezentacije), moramo dodati fragile opciju na svaki frame (okvir)
		\end{alertblock}
	\end{frame}
	
	\begin{frame}[fragile]
		\frametitle{Kod unutar dokumenta}
		\begin{block}{Veći dijelovi koda}
			Koristimo lstlisting okruženje (environment): \verb|\begin{lstlisting} Neki kod \end{lstlisting}|
		\end{block}
		\begin{block}{Kraći kod}
			Ako želimo integrirati kod s ostatkom teksta koristimo: \verb|\lstinline{Neki kod}|
		\end{block}
		\begin{block}{Vanjski kod}
		Ako smo izvorni kod napisali u drugoj datoteci, praktično je uvesti ga i ažurirati ponovnim kompajliranjem LaTex dokumenta u slučaju izmjena.
		\verb|\lstinputlisting{izvorni_kod.ekstenzija}|
		\end{block}
	\end{frame}
	\begin{frame}[fragile]
		\frametitle{Postavke i stilovi}
		Postavke definiramo na više načina:
		\begin{itemize}
			\begin{item}
			U istoj naredbi u kojoj uključujemo datoteku s kodom: \verb|\lstinputlisting[postavka1, postavka2]{izvorni_kod.py}|
			\end{item}
			\begin{item}			
			U naredbi u kojoj započinjemo listings okruženje: \verb|\begin{lstlisting}[postavka1, postavka2]|
			\end{item}
			\begin{item}
Unutar naredbe \verb|\lstset{}| koja se može nalaziti bilo gdje u kodu, prije ili poslije \verb|\begin{document}|, postavke se primjenjuju na sav kod do iduće naredbe koja ih mijenja
			\end{item}
		\end{itemize}
		\begin{block}{Definiranje više stilova}
				Naredbom \verb|\lstdefinestyle{ime_stila}{postavka1, postavka2}| definiramo stil koji možemo koristiti više puta pomoću \verb|\lstset{sytle=ime_stila}|
		\end{block}
	\end{frame}
	\begin{frame}[fragile]
		\frametitle{Dodatni paketi}
		\begin{block}{Paket xcolor i color}
		Paketi xcolor ili color su potrebni za neke od postavki obojanja teksta ili pozadine. Uključujemo ih u dokument pomoću
\verb|\usepackage{color}| ili \verb|\usepackage{xcolor}|.
Osim standradnih boja možemo definirati svoje nijanse: \verb|\definecolor{mycolor}{rgb}{0.7,0.6,0.8}|
Boje pozivamo naredbom \verb|\color{name}|.
	\end{block}
	\end{frame}
	
	\begin{frame}[fragile]
		\frametitle{Osnovne postavke (1. dio)}
		\begin{description}
			\item[language=Python] Ako ne definiramo programski jezik, nemamo dodatne opcije procesiranja koje vrijede za njega
			\item[firstline=37, lastline=55] Raspon linija koji će se prikazati (uključivo s navedenim), definiran samo firstline = linije prije odabrane, samo lastline = linije nakon odabrane
			\item[backgrouncolor=] Odabir boje pozadine, npr. \verb|\color{white}|, trebao bi biti poslijednji argument, potreban paket xcolor ili color
			\item[caption=tekst] Opis koji se prikazuje uz kod
			\item[captionpos=b] Pozicija opisa, b = dno, t = vrh, h = trenuta lokacija u dokumentu, p = posebna stranica, isto kao za float elemente sliku ili tablicu
			\item[morekeywords=] Dodajemo još ključnih riječi koje se označuju, navodimo ih unutar vitičastih zagrada odvojene zarezima
			\item[deletekeywords=] Brišemo ključne riječi jezika, navodimo ih unutar vitičastih zagrada odvojene zarezima
			\end{description}
	\end{frame}
\begin{frame}[fragile]
		\frametitle{Osnovne postavke (2. dio)}
		\begin{description}
			\item[keywordstyle=] Mijenjamo boju i veličinu ključnih riječi
			\item[basicstyle=] Mijenjamo boju i veličinu običnog koda, koji nije ključna riječ ni komentar ni string
			\item[stringstyle=] Mijenjamo boju i veličinu stringova, riječi
			\item[commentstyle=] Mijenjamo boju i veličinu komentara
			\item[numberstyle=] Stil brojeva linije
			\item[numbers=left] Broj linije koda će biti lijevo (left), desno (right), ili izostavljen (none)
			\item[numbersep=5pt] Razmak između koda i broja linija
			\item[stepnumber=2] Svaka n – ta linija je numerirana, razmak između numeriranih linija
			\item[listoflistings] Ispisuje popis captiona listings okruženja u dokumentu
			\item[escapeinside=] Privremeno izlazimo iz listings okruženja, između dva znaka navedena svaki u zasebnim vitičastim zagradama tekst se procesira kao Latex naredbe
			\end{description}
		\end{frame}
		\begin{frame}[fragile]
		\frametitle{Osnovne postavke (3. dio)}
		\begin{description}
			\item[showstringspaces=true] Prikazuje razmake unutar stringova pomoću znaka breve, koji izgleda kao donja polovica kruga, ako ne postavimo je false
			\item[showspaces=true] Prikazuje sve razmake pomoću znaka breve, koji izgleda kao donja polovica kruga, poništava showstringsspaces, ako ne postavimo je false
			\item[showtabs=true] Prikazuje tabulatore pomoću strelice usmjerene slijeva nadesno, ako ne postavimo je false
			\item[tabsize=2] Veličina tabulatora u razmacima
			\item[title=] Dodajemo naslov kodu
			\item[xleftmargin=, xrightmargin=] Veličina dodatne lijeve ili desne margine
			\item[frame=] none - nema, default, leftline - lijeva linija, topline - gornja linija, bottomline - donja linija, lines - linija na vrhu i dnu, single - okvir, shadowbox - osjenčana kutija
			
			\item[frameround] -  

			\end{description}
	\end{frame}
\end{document}
